\documentclass[12pt]{article}

\usepackage{snow-preamble}

\newtheorem{prop}{Proposition}

%%%%%%%%%%%%%%%%%%%%%%%%%%%%%%%%%%%%
\begin{document}
	\maketitle
	\tableofcontents

	\newpage

	%% body start

	\section{Objective}
		\begin{prop} \label{prop:1}
			If a FWP and a weighted matrix are to play the same game of checkers with no training phase the FWP
			will play more efficiently.
		\end{prop}

	% what is an FWP
	\section{}

	% how does it apply to graph theory
	\section{}

	% why are they usefull
	\section{Literature Review}

	% application
	\section{Methods}
		\subsection{American checkers rules} \label{sec:rules}
			\begin{itemize}
				\item played on a board perfectly latticed board
				\item pieces can move forward unless capturing
				\item to capture a piece must jump another piece (of opposite loyalty) diagonally forward
				\item pieces that reach the opposite side promote to "kings"
				\item "kings" can capture forward or backward
			\end{itemize}
		\subsection{Experiment states}
			\indent The different iterations serve to symbolize different stages of internal development. I devise
			the different stages by the amount of weakenings placed on the problem. We begin with a proposition
			\ref{prop:1}, then create stages by applying weakenings to the problem such that proving these
			weakened stages will bring us closer to proving our original proposition\footnote{Such is the format
			for any complex direct proof}.


	%% body finish

	\nocite{*}

	\newpage % bib
	\backmatter
	\addcontentsline{toc}{section}{References}
	\printbibliography

	\newpage % index
	\printindex

	\newpage % appendices
	\appendix

\end{document}
